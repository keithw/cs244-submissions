\documentclass[12pt]{article}

%\usepackage[dvips]{graphics,color}
\usepackage{fullpage}
%\usepackage{amsfonts}
\usepackage{amssymb}
\usepackage{amsmath}
%\usepackage{latexsym}
\usepackage{enumerate}
\usepackage{fancybox}
\usepackage{float}
\usepackage{tikz}
\usepackage{gensymb}
\usepackage{units}
\usepackage{ifthen}
\usetikzlibrary{calc}
\usetikzlibrary{automata,positioning}
\pagenumbering{gobble}

\usepackage{graphicx}
\graphicspath{{./}}
\usepackage{wrapfig}
\DeclareGraphicsExtensions{.png,.jpg}

\setlength{\parskip}{1pc}
\setlength{\topmargin}{-1pc}
\setlength{\textheight}{9in}

\begin{document}
\begin{center}
{\Large CS244 PA2 Writeup}
\begin{center}
Gus Liu, Eli Berg - Spring 2016 \\
sunetids: gusliu, ejberg
\end{center} 
\end{center}

\section*{Warmup exercise A}
	
	
\section*{Warmup exercise B}
AIMD performed well in terms of throughput but poorly for delay. We chose the following constants:\\
Timeout = 80ms, 
SSThresh = 13, 
Multiplicative decrease factor = 2.\\
Our best score was $\frac{3.81}{0.295} = 12.92$. \\

This makes sense because AIMD waits for a "timeout" to occur before cutting the window. Moreover, the multiplicative decrease gives us less granularity in our control over the window. 
	
\section*{Warmup exercise C}
	This delay-triggered scheme doesn't work very well. Below are our various results from runs.
\\
\\
Linear increase, exponential decrease, thresh = 75: $\frac{1.55}{0.205}$ = 7.56 \\
Linear increase, linear decrease, thresh =  75: $\frac{2.46}{0.479}$ = 5.14 \\
Exponential increase, exponential decrease, thresh =  75: $\frac{2.46}{0.220}$ = 11.18 \\
Exponential increase, exponential decrease, thresh =  100: $\frac{2.90}{0.289}$ = 10.03 \\
Exponential increase, exponential decrease, thresh =  85: $\frac{2.58}{0.229}$ = 11.26 \\
\\
Exponential increase and decrease with a threshold of 85ms worked the best. We believe this is because it allows the window to more rapidly adjust to changing network conditions compared to linear adjustments. Also, a threshold too low doesn't optimize for throughput, while a threshold too high adjusts too late. 

\section*{Warmup exercise D}

	
		
\end{document}




